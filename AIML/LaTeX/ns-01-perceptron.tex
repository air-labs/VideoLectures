\documentclass[24pt,pdf,hyperref={unicode}]{beamer}
\usepackage[utf8]{inputenc}
\usepackage[russian]{babel}
\usepackage{graphics}
\usepackage{amssymb}
\usepackage{xstring}
\usepackage{multirow}
\usepackage{tikz}
\usepackage[all]{xy}


\newcommand{\dd}[2]{\frac{\partial #1}{\partial #2}}
\begin{document}
\tikzstyle{neu}=[circle,fill=blue!50];


\section{Персептрон}

\begin{frame}\frametitle{Персептрон}
\begin{columns}
\column{0.5\textwidth}
$$
\xymatrix
{
x_1 \ar@{->}[rd]^{w_1} \\
x_2 \ar@{->}[r]^{w_2} & *++[Fo]{f} \ar@{->}[r] & y\\
x_3 \ar@{->}[ru]_{w_3} \\
}
$$
\column{0.5\textwidth}
$$
y=f\left(\sum_{i=1}^{n}w_ix_i\right)
$$\\[1cm]
$$
f(x)={\rm sign}(x)
$$
\end{columns}
\end{frame}


\begin{frame}\frametitle{Вес активации}
\begin{columns}
\column{0.5\textwidth}
$$
\xymatrix
{
x_1 \ar@{->}[rd]^{w_1} \\
x_2 \ar@{->}[r]^{w_2} & *++[Fo]{f} \ar@{->}[r] & y\\
x_3 \ar@{->}[ru]_{w_3} \\
}
$$
\column{0.5\textwidth}
$$
\xymatrix
{
x_1 \ar@{->}[rd]^{w_1} & x_0\equiv 1 \ar@{->}[d]^{w_0}\\
x_2 \ar@{->}[r]^{w_2} & *++[Fo]{f} \ar@{->}[r] & y\\
x_3 \ar@{->}[ru]_{w_3} \\
}
$$
\end{columns}
\end{frame}

\begin{frame}\frametitle{Реализация конъюнкции}
\begin{columns}
\column{0.5\textwidth}
$$
\begin{array}{c c | c}
x_1 & x_2 & x_1\wedge x_2 \\
\hline
0 & 0 & 0 \\
0 & 1 & 0 \\
1 & 0 & 0 \\
1 & 1 & 1 \\
\end{array}
$$
\column{0.5\textwidth}
$$
\left\{
\begin{array}{l l l l l l}
 w_0       & < & 0 \\
 w_0 + w_1 & < & 0 \\
 w_0 + w_2 & < & 0 \\
 w_0 + w_1 + w_2 & > & 0 \\
\end{array}
\right.
$$\\[1cm]
\uncover<2>{
$$
\begin{array}{l}
 w_0 = 3 \\
 w_1 = 2 \\
 w_2 = 2 \\
\end{array}
$$
}
\end{columns}
\end{frame}


\begin{frame}\frametitle{Реализация дизъюнкции}
\begin{columns}
\column{0.5\textwidth}
$$
\begin{array}{c c | c}
x_1 & x_2 & x_1\vee x_2 \\
\hline
0 & 0 & 0 \\
0 & 1 & 1 \\
1 & 0 & 1 \\
1 & 1 & 1 \\
\end{array}
$$
\column{0.5\textwidth}
$$
\left\{
\begin{array}{l l l l l l}
 w_0       & < & 0 \\
 w_0 + w_1 & > & 0 \\
 w_0 + w_2 & > & 0 \\
 w_0 + w_1 + w_2 & > & 0 \\
\end{array}
\right.
$$\\[1cm]
$$
\begin{array}{l}
 w_0 = 1 \\
 w_1 = 2 \\
 w_2 = 2 \\
\end{array}
$$
\end{columns}
\end{frame}

\begin{frame}\frametitle{Геометрическая интерпретация}
\uncover<+->{}
\begin{tikzpicture}[y=4cm, x=4cm]
\draw[->] (-0.1,0) -- (1.5,0) node[below]{$x_1$};
\draw[->] (0,-0.1) -- (0,1.5) node[left]{$x_2$};
\draw[fill=black] (0,0) circle(3pt);
\draw[fill=black] (0,1) circle(3pt);
\draw[fill=black] (1,0) circle(3pt);
\draw[fill=white] (1,1) circle(3pt);

\uncover<+->{
\draw (0.1,1.4) -- (1.4,0.1);
\draw[->] (0.75,0.75) -- (0.85,0.85) node[right] {$(w_1,w_2)$};
}
\end{tikzpicture}
\end{frame}

\begin{frame}\frametitle{Обучение персептрона}
Тут про советчиков, м.б. тоже в PP
\end{frame}

\end{document}