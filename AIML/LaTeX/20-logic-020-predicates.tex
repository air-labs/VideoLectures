\documentclass[24pt,pdf,hyperref={unicode}]{beamer}
\usepackage[utf8]{inputenc}
\usepackage{aiml}

\begin{document}


\begin{frame}\frametitle{Дедуктивные рассуждения}
\uncover<+->{
\begin{tabular}{l l}
 & Все страны Южной Америки -- республики \\
 & Бразилия -- страна Южной Америки\\
 \hline
$\therefore$ & Бразилия -- республика \\
\end{tabular}\\[2cm]
}
\uncover<+->{
{\large
$$
\frac{A, B}{?}
$$
}
}
\end{frame}

\begin{frame}\frametitle{Модель}
\uncover<+->{
$$
\mathfrak{M}=(M,P_1,\ldots,P_n)
$$
}
\uncover<+->{
$$
P_i:M^{k_i}\rightarrow\{0,1\}
$$
}
\end{frame}

\usetikzlibrary{trees}

\begin{frame}\frametitle{Модель}
\begin{columns}
\column{0.5\textwidth}
Генеологическое древо

\column{0.5\textwidth}

$P(x,y)$, $C(x,y)$\\[1cm]

\begin{itemize}
\item<+-> $P(Tywin,Cercei)$

\item<+-> $C(Tommen,Cercei)$

\item<+-> $C(Tyrion,Cercei)$

\item<+-> $\forall x,y\ P(x,y)\rightarrow C(y,x)$

\item<+-> $\exists u,x,y,z\ C(u,x)\wedge C(u,y)\wedge C(x,z)\wedge C(y,z)$

\item<+-> $\forall x \exists y\ P(y,x)$
\end{itemize}
\end{columns}
\end{frame}

\begin{frame}\frametitle{Модель}
\uncover<+->{
\begin{center}
$\mathfrak{N}=(\mathbb{N},P,S)$

$Eq(x,y)\Leftrightarrow x=y$

$S(x,y,z) \Leftrightarrow x+y=z$

$P(x,y,z) \Leftrightarrow xy=z$
\end{center}
}
\uncover<+->{
\begin{tabular}{p{4cm} p{5cm}}
Свойство единицы & $\forall x\ P(x,1,x) \wedge P(1,x,x)$ \\
Коммутативность\newline сложения & $\forall x,y,z\ S(x,y,z)\rightarrow S(y,x,z)$\\
Отображение & $\forall x,y,z,u\ (P(x,y,u)\wedge P(x,y,v))\rightarrow Eq(u,v)$ \\
Тотальное отображение & $\forall x,y\ \exists z\ S(x,y,z)$ \\
Вычитание & $\forall x,y\ \exists z\ S(x,z,y)$ \\
\end{tabular}
}
\end{frame}

\begin{frame}\frametitle{Вывод в логике предикатов}

$$
\begin{array}{l l}
& (\forall x\ P(x))\rightarrow(\forall x\ Q(x)) \\
& (\forall x\ P(x)) \\
\hline
& A=(\forall x\ P(x)),\ B=(\forall x\ Q(x)) \\
\therefore & \forall x\ Q(x)
\end{array}
$$\\[1cm]
$$
\begin{array}{l l}
& \forall x\ P(x)\rightarrow Q(x) \\
& \forall x\ P(x) \\
\hline
\therefore & \forall x\ Q(x)
\end{array}
$$
\end{frame}

\begin{frame}\frametitle{Приведение к ПНФ}
\begin{enumerate}
\item<1-> Оставить только $\wedge$, $\vee$, $\neg$
\item<2-> Протаскивание отрицаний: законы де Моргана и

$$
\neg(\forall x\ P(x))\Leftrightarrow \exists x\ \neg P(x)
$$
$$
\neg(\exists x\ P(x))\Leftrightarrow \forall x\ \neg P(x)
$$
\item<3-> { Вытаскивание кванторов }

\begin{center}

\uncover<4->{$
(\forall x\ P(x))\wedge(\forall x\ Q(x))\Leftrightarrow \forall x\ (P(x)\wedge Q(x))
$}

\only<5>{$
(\exists x\ P(x))\wedge(\exists x\ Q(x))\Leftrightarrow \exists x\ (P(x)\wedge Q(x) )
$}
\uncover<6->{$
(\exists x\ P(x))\wedge(\exists x\ Q(x))\Leftrightarrow \exists x\exists y\ (P(x)\wedge Q(y))
$}


\uncover<7->{$
(\exists x\ P(x))\vee(\exists x\ Q(x))\Leftrightarrow \exists x\ (P(x)\vee Q(x))
$}


\uncover<8->{$
(\forall x\ P(x))\vee(\forall x\ Q(x))\Leftrightarrow \forall x\forall y\ (P(x)\vee Q(x))
$}
\end{center}
\item<9-> Результат: предваренная нормальная форма

$$
\forall x_1 \forall x_2 \exists x_3 \forall x_4\ F
$$
\end{enumerate}

\end{frame}

\end{document}
