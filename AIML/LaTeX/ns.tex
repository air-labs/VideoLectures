\documentclass[24pt,pdf,hyperref={unicode}]{beamer}
\usepackage[utf8]{inputenc}
\usepackage[russian]{babel}
\usepackage{graphics}
\usepackage{amssymb}
\usepackage{xstring}
\usepackage{multirow}
\usepackage[all]{xy}


\newcommand{\dd}[2]{\frac{\partial #1}{\partial #2}}
\begin{document}

\section{Персептрон}

\begin{frame}\frametitle{Персептрон}
\begin{columns}
\column{0.5\textwidth}
$$
\xymatrix
{
x_1 \ar@{->}[rd]^{w_1} \\
x_2 \ar@{->}[r]^{w_2} & *++[Fo]{f} \ar@{->}[r] & y\\
x_3 \ar@{->}[ru]_{w_3} \\
}
$$
\column{0.5\textwidth}
$$
y=f\left(\sum_{i=1}^{n}w_ix_i\right)
$$\\[1cm]
$$
f(x)={\rm sign}(x)
$$
\end{columns}
\end{frame}


\begin{frame}\frametitle{Вес активации}
\begin{columns}
\column{0.5\textwidth}
$$
\xymatrix
{
x_1 \ar@{->}[rd]^{w_1} \\
x_2 \ar@{->}[r]^{w_2} & *++[Fo]{f} \ar@{->}[r] & y\\
x_3 \ar@{->}[ru]_{w_3} \\
}
$$
\column{0.5\textwidth}
$$
\xymatrix
{
x_1 \ar@{->}[rd]^{w_1} & x_0\equiv 1 \ar@{->}[d]^{w_0}\\
x_2 \ar@{->}[r]^{w_2} & *++[Fo]{f} \ar@{->}[r] & y\\
x_3 \ar@{->}[ru]_{w_3} \\
}
$$
\end{columns}
\end{frame}

\begin{frame}\frametitle{Реализация конъюнкции}
\begin{columns}
\column{0.5\textwidth}
$$
\begin{array}{c c | c}
x_1 & x_2 & x_1\wedge x_2 \\
\hline
0 & 0 & 0 \\
0 & 1 & 0 \\
1 & 0 & 0 \\
1 & 1 & 1 \\
\end{array}
$$
\column{0.5\textwidth}
$$
\left\{
\begin{array}{l l l l l l}
 w_0       & < & 0 \\
 w_0 + w_1 & < & 0 \\
 w_0 + w_2 & < & 0 \\
 w_0 + w_1 + w_2 & > & 0 \\
\end{array}
\right.
$$\\[1cm]
\uncover<2>{
$$
\begin{array}{l}
 w_0 = 3 \\
 w_1 = 2 \\
 w_2 = 2 \\
\end{array}
$$
}
\end{columns}
\end{frame}


\begin{frame}\frametitle{Реализация дизъюнкции}
\begin{columns}
\column{0.5\textwidth}
$$
\begin{array}{c c | c}
x_1 & x_2 & x_1\vee x_2 \\
\hline
0 & 0 & 0 \\
0 & 1 & 1 \\
1 & 0 & 1 \\
1 & 1 & 1 \\
\end{array}
$$
\column{0.5\textwidth}
$$
\left\{
\begin{array}{l l l l l l}
 w_0       & < & 0 \\
 w_0 + w_1 & > & 0 \\
 w_0 + w_2 & > & 0 \\
 w_0 + w_1 + w_2 & > & 0 \\
\end{array}
\right.
$$\\[1cm]
$$
\begin{array}{l}
 w_0 = 1 \\
 w_1 = 2 \\
 w_2 = 2 \\
\end{array}
$$
\end{columns}
\end{frame}

\begin{frame}\frametitle{Геометрическая интерпретация}
в PP
\end{frame}

\begin{frame}\frametitle{Обучение персептрона}
Тут про советчиков, м.б. тоже в PP
\end{frame}




\section{Многослойный персептрон}

\begin{frametitle}\frametitle{Функция XOR}
 
\end{frametitle}

\begin{frametitle}\frametitle{Функция XOR}
 
\end{frametitle}

\begin{frametitle}\frametitle{Многослойный персептрон}
 
\end{frametitle}


\begin{frame}\frametitle{Постановка задачи}
{\bf Дано:}
\begin{tabular}{p{4cm} p{6cm}}
 $\mathcal{I}=(I_1,\ldots,I_k)$ & входные вектора размерности $n$\\[0.1cm]
 $\mathcal{A}=(A_1,\ldots,A_k)$ & правильные выходные вектора размерности $m$\\[0.1cm]
 $(\mathcal{I},\mathcal{A})$ & обучающая выборка  \\[0.1cm]
 $N(W,I)$ & функция, соответствующая нейронной сети \\[0.1cm]
 $O_i=N(W,I_i)$ & ответ нейронной сети, вектор размерности $m$ \\[0.1cm]
$E(O_i,A_i)$ & \\
$=\sum_{j=1}^{m} (O_i[j]-A_i[j])^2 $ & функция ошибки \\
 \end{tabular}\\[1cm]
{\bf Найти:}
вектор $W$ такой, что $\sum_{i=1}^k E(N(W,I_i)-A_i)\rightarrow \min$
\end{frame}


\begin{frame}\frametitle{Обучение онлайн}
Решим задачу для одной пары $(I,A)$ \\[2cm]

В этом случае $E(N(W_i,I)-A)$ является функцией от вектора весов $E=E(W)$. 
\end{frame}

\begin{frame}\frametitle{Частные производные}
Функция $n$ переменных:
$$
F:\mathbb{R}^n\rightarrow\mathbb{R} 
$$
$$
F(x_1,\ldots,x_n)
$$
Частная производная по $i$-й переменной:
$$
\dd{F}{x_i}(x_1,\ldots,x_n)=
$$
$$
=\lim_{\varepsilon\rightarrow 0}\frac{F(x_1,x_2,\ldots,x_i+\varepsilon,\ldots,x_n)-F(x_1,x_2,\ldots,x_i,\ldots,x_n)}{\varepsilon}
$$
$$
\dd{F}{x_i}:\mathbb{R}^n\rightarrow\mathbb{R}
$$

\end{frame}
\uncover<+->{}
\begin{frame}\frametitle{Частные производные}
$$
F(x,y,z,u)=x^3+y^u+\sin z^2u^3
$$
$$
\begin{array}{l l}
\dd{F}{x} = & \uncover<+->{3x^2} \\
\dd{F}{y}= & \uncover<+->{uy^u-1} \\
\dd{F}{z}= & \uncover<+->{(-\cos z^2u^3)(u^32z)}
\end{array}
$$
\end{frame}

\begin{frame}\frametitle{Производная сложной функции}
$$
F=F(x_1,\ldots,x_n)
$$
$$
G_i=G_i(y_1,\ldots,y_m)
$$
$$
H(y_1,\ldots,y_n)=F(G_1(y_1,\ldots,y_m),\ldots, G_n(y_1,\ldots,y_n))
$$
$$
\dd{H}{y_i}=\sum_{j=1}^{n}\dd{H}{G_j}\dd{G_j}{y_i}
$$
\end{frame}

\begin{frame}\frametitle{Производная сложной функции}
\uncover<+->{}
$$
F(x_1,\ldots,x_n)=\sum_{k=1}{n} a_ix_i
$$
$$
G_i(y_1,\ldots,y_m)=\sum_{k=1}{m}y_k^i
$$
$$
H(y_1,\ldots,y_n)=F(G_1(y_1,\ldots,y_m),\ldots, G_n(y_1,\ldots,y_n))
$$
\uncover<+->{
$$
\dd{F}{G_i}=a_i,\ \dd{G_i}{y_j}=iy_j^{i-1}
$$
}
$$
\dd{H}{y_j}=\uncover<+->{\sum_{i=1}^{n}\dd{H}{G_i}\dd{G_i}{y_j}=\sum_{i=1}^{n} a_iiy_j^{i-1}}
$$

\end{frame}

\begin{frame}\frametitle{Градиент}
\uncover<+->{}
$$
\nabla F = \left(\dd{F}{x_1},\dd{F}{x_2},\ldots,\dd{F}{x_n}\right)
$$
$$
\nabla F : \uncover<+->{\mathbb{R}^n\rightarrow\mathbb{R}^n}
$$
\end{frame}

\begin{frame}\frametitle{Градиентный спуск}

\end{frame}

\begin{frame}\frametitle{Градиентный спуск}
 

\end{frame}





\end{document}